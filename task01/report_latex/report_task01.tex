\documentclass{article}

\usepackage{hyperref}
\usepackage{amsmath}

% Page layout
\hoffset -0in
\voffset -1in
\oddsidemargin 0in
\textheight 10in
\textwidth 6.3in

\setlength{\parindent}{0pt}
\pagestyle{empty}

\begin{document}
	\section*{COMPUTATIONAL METHODS IN MECHANICS: Task 1}
	Vesa-Ville Hurskainen, 19 Jan 2018
	
	\section*{Introduction}
	This is a report on the first assignment of the course \textit{Computational Methods in Mechanics}, concerning the trapezoidal integration rule. A MATLAB script with an implementation of the rule (\texttt{integral\_trapezoid.m}) was provided and six tasks involving the script were given. The tasks are as follows:
	
	\begin{enumerate}
		\setlength\itemsep{0pt}
		\item To test that the script works correctly.
		\item To use GitHub.
		\item To compare the script with built-in MATLAB procedure.
		\item To test the debugger.
		\item To speed up the script.
		\item To use the script to compute a 2D integral.
	\end{enumerate}
	
	\section*{Methods}
	In order to complete Task 5, an optimized script (\texttt{integral\_trapezoid\_optimized.m}) was written, employing vectorization to speed up computation. In addition, two test scripts (\texttt{test\_integral\_1D.m} and \texttt{test\_integral\_2D.m}) were written in order to test and compare the various different types of integration functions in one- and two-dimensional cases. There, the MATLAB profiler was used to investigate computation times and the MATLAB debugger to find errors. A function was chosen to test the integration procedures in each case. The functions, along with their analytical definite integrals, are as follows:
	\begin{align}
	\int_{0}^{1} \textrm{sin}(x)~dx &= 1-\textrm{cos}(1) \\
	\int_{0}^{1} \int_{0}^{1} \textrm{sin}(x) + \textrm{cos}(y)~dx~dy &= 1 + \textrm{sin}(1) - \textrm{cos}(1)
	\end{align}
	
	\section*{Results}
	All employed scripts can be found in a \href{https://github.com/VesaVilleHurskainen/cmim2018}{GitHub repository}. Numerical results are presented in Table~\ref{tab:results}.
	 \begin{table}[ht]
	 	\def\arraystretch{1.3}
		\begin{center}
			\begin{tabular}{|c|c|c|c||c|c|c|}
				\hline
				\multicolumn{4}{|c||}{$\int_{0}^{1} \textrm{sin}(x) dx$} & \multicolumn{3}{c|}{$\int_{0}^{1} \int_{0}^{1} \textrm{sin}(x) + \textrm{cos}(y) dx~dy$}\\
				\hline
				Integration                    & Steps  & Result    & Avg. time & Integration                        & Steps  & Result \\
				\hline
				Analytical                     &        & 0.4596977 &           & Analytical                &                    & 1.3011687 \\
				MATLAB \texttt{trapz} & $10^6$ & 0.4596977 & 35 ms               & MATLAB \texttt{integral2} & (defaults) & 1.3011687 \\
				Original script                & $10^6$ & 0.4596977 & 1100 ms   & Original script           & 1000 / dim & 1.3011687 \\
				Optimized script               & $10^6$ & 0.4596977 & 20 ms &&&\\
				\hline
			\end{tabular}
			\caption{Comparison of numerical results yielded by test scripts.}
			\label{tab:results}
		\end{center}
	\end{table}
	\vspace{-1cm}

	\section*{Analysis}
	Tasks 2 and 4 were completed during coding and the completion of Tasks 1 and 3 can be verified by comparing the results presented in Table~\ref{tab:results}. The script appears to work correctly, since it produces results that coincide with the analytical results in the one-dimensional case. However, the script is unoptimized and therefore significantly slower than than the MATLAB function \texttt{trapz}. On the other hand, the optimized script is even faster than the built-in script when using vectorized functions, which completes Task 5. Finally, as Table~\ref{tab:results} shows, in the two-dimensional case the same result was gained from both the script and the built-in MATLAB function \texttt{integral2}, which completes Task 6. Therefore, all tasks are complete.\\
\end{document}