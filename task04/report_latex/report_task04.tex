\documentclass{article}

\usepackage{hyperref}
\usepackage{amsmath}
\usepackage{graphicx}
\usepackage{subcaption}
\usepackage{epstopdf}
\usepackage{bm}

\usepackage{times}

% Page layout
\hoffset -0in
\voffset -1in
\oddsidemargin 0in
\textheight 9.3in
\textwidth 6.3in

\setlength{\parindent}{0pt}
%\setlength{\intextsep}{10pt}
%\pagestyle{empty}

\graphicspath{{figures/}}

\begin{document}
	\section*{COMPUTATIONAL METHODS IN MECHANICS: Task 4}
	Vesa-Ville Hurskainen, 14 Feb 2018\\
	\href{https://github.com/VesaVilleHurskainen/cmim2018}{GitHub repository}

	\section*{Introduction}
	This is a report of the fourth assignment of the course \textit{Computational Methods in Mechanics}, concerning numerical integration. The assignment consists of six tasks, which are as follows:
	
	\begin{enumerate}
		\setlength\itemsep{0pt}
		\item To write the equations of motion for a three-mass-three-spring system.
		\item To numerically calculate the system's natural frequencies.
		\item To investigate how removing the first spring affects the natural frequencies.
		\item To solve the system with nonzero initial conditions using four solvers (odeRK4, odeSIE, ode45, ode15s) and compare accuracy and
		efficiency.
		\item To repeat the computations using nodal coordinates and compare the results with the original system.
		\item To write a report.
	\end{enumerate}

	\section*{Methods}
	Using Newton's second law, we can derive the following system of equations of motion for the three-mass system: 
	\begin{equation}
		\begin{aligned}
		m_1 \ddot{x}_1 & = - x_1 k_1 + (x_2 - x_1) k_2 \\
		m_2 \ddot{x}_2 & = - (x_2 - x_1) k_2 + (x_3 - x_2) k_3\\
		m_3 \ddot{x}_3 & = - (x_3 - x_2) k_3
		\end{aligned}
	\end{equation}
	Transforming the system of linear equations into matrix form, we get the matrix equation:
	\begin{equation}
		\mathbf{M} \ddot{\bm{x}} + \mathbf{K} \bm{x} = \bm{0}
	\end{equation}
	where the mass matrix $\mathbf{M}$ and stiffness matrix $\mathbf{K}$ are written as:
	\begin{align}
		\mathbf{M} &= \text{diag}(m_1, m_2, m_3) \\
		\mathbf{K} &= \begin{bmatrix}
		k_1 + k_2 & -k_2 & 0 \\
		-k_2 & k_2 + k_3 & - k_3 \\
		0 & -k_3 & k_3
		\end{bmatrix}
	\end{align}
	
	\section*{Results}
	The numerically calculated natural frequencies of the original and modified system are presented in Table~\ref*{tab:natural_frequencies}.
	
	\begin{table}[htb]
		\centering
		\begin{tabular}{|r|r|}
			\hline
			Original system & Modified: $k_1 = 0$ \\
			\hline
			51.08 rad/s & 0 rad/s \\
			117.5 rad/s & 94.38 rad/s \\
			166.6 rad/s & 158.9 rad/s \\
			\hline
		\end{tabular}
		\caption{Numerically computed natural frequencies of the system.}
		\label{tab:natural_frequencies}
	\end{table}

	\section*{Analysis}
	As Table~\ref*{tab:natural_frequencies} shows, the first natural frequency of the system goes to zero when the first spring is removed. This was expected, since doing so removes all stiffness constricting the system's movement in relation to the reference frame. The other two natural frequencies are also lowered; an expected result of reduced stiffness.
	
	\clearpage
	\section*{Secret extra page: integrator testing}
	
\end{document}