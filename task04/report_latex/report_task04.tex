\documentclass{article}

\usepackage{hyperref}
\usepackage{amsmath}
\usepackage{graphicx}
\usepackage{subcaption}
\usepackage{epstopdf}
\usepackage{bm}

\usepackage{times}

% Page layout
\hoffset -0in
\voffset -1in
\oddsidemargin 0in
\textheight 9.3in
\textwidth 6.3in

\setlength{\parindent}{0pt}
%\setlength{\intextsep}{10pt}
%\pagestyle{empty}

\graphicspath{{figures/}}

\begin{document}
	\section*{COMPUTATIONAL METHODS IN MECHANICS: Task 4}
	Vesa-Ville Hurskainen, 14 Feb 2018\\
	\href{https://github.com/VesaVilleHurskainen/cmim2018}{GitHub repository}

	\section*{Introduction}
	This is a report of the fourth assignment of the course \textit{Computational Methods in Mechanics}. The assignment consists of ... tasks, which are as follows:
	
	\begin{enumerate}
		\setlength\itemsep{0pt}
		\item To use GitHub.
		\item ...
	\end{enumerate}

	\section*{Methods}
	Using Newton's second law, we can derive the following system of equations of motion for the three-mass system: 
	\begin{equation}
		\begin{aligned}
		m_1 \ddot{x}_1 & = - x_1 k_1 + (x_2 - x_1) k_2 \\
		m_2 \ddot{x}_2 & = - (x_2 - x_1) k_2 + (x_3 - x_2) k_3\\
		m_3 \ddot{x}_3 & = - (x_3 - x_2) k_3
		\end{aligned}
	\end{equation}
	Transforming the system of linear equations into matrix form, we get the matrix equation:
	\begin{equation}
		\mathbf{M} \ddot{\bm{x}} + \mathbf{K} \bm{x} = \bm{0}
	\end{equation}
	where the mass matrix $\mathbf{M}$ and stiffness matrix $\mathbf{K}$ are written as:
	\begin{align}
		\mathbf{M} &= \text{diag}(m_1, m_2, m_3) \\
		\mathbf{K} &= \begin{bmatrix}
		k_1 + k_2 & -k_2 & 0 \\
		-k_2 & k_2 + k_3 & - k_3 \\
		0 & -k_3 & k_3
		\end{bmatrix}
	\end{align}
	
	\section*{Results}
	...

	\section*{Analysis}
	...
	
\end{document}